\subsection{REST API Summary}

%describe the REST API. If needed, add a few lines of text here, describing the content of the table.

For this homework, we implemented REST APIs on the paths "/rest/causes/" and "/rest/tags/" for the Causes and Tags resources. These resources are used as properties of the Events resource to facilitate retrieval and help categorization. \\
Specifically the REST API is used in the "/tag-search", "/cause-search" and "/cause-create" pages accessible to administrators to manage these resources.
To match future development needs or just uniform the methods of information retrieval between pages we will probably extend the area of coverage of REST APIs.
The following table shows the structure we followed to implement "/rest/causes/".

\begin{longtable}{|p{.30\columnwidth}|p{.10\columnwidth} |p{.25\columnwidth}|p{.25\columnwidth}|} 
\hline
\textbf{URI} & \textbf{Method} & \textbf{Description} & \textbf{Filter} \\\hline
/rest/causes/ & GET & It returns a list of all Causes in the database & Behind AdminFilter (only accessible to tier 4 users)\\\hline
/rest/causes/id/*id* & GET & It returns the Cause with the corresponding *id*, if *id* is empty it returns all Causes & Behind AdminFilter (only accessible to tier 4 users)\\\hline
/rest/causes/srch/*subCause* & GET & It returns the Causes that contain the string *subCause* in the name field & Behind AdminFilter (only accessible to tier 4 users)\\\hline
/rest/causes/ & POST & It creates a new Cause contained in the request, inserts it into the database and returns it & Behind AdminFilter (only accessible to tier 4 users)\\\hline
/rest/causes/id/*id* & PUT & It updates the Cause with the corresponding id with the values contained in the request and returns it & Behind AdminFilter (only accessible to tier 4 users)\\\hline
/rest/causes/id/*id* & DELETE & It deletes the Cause with the corresponding id and returns it & Behind AdminFilter (only accessible to tier 4 users)\\\hline
\caption{REST API for the Causes resource}
\label{tab:termGlossary}
\end{longtable}

\pagebreak