\section{Presentation Logic Layer}

%What pages will be present in your project? briefly indicate how your web site will be organized

\lipsum[1]

\subsection{Signup page}
\begin{figure}[H]
    \centering
    \includegraphics[width=0.6\textwidth]{images/signup.png}
    \caption{Sign up page}
    \label{fig:signup page}
\end{figure}
Users can sign up just with mail and password. They will not have access only to their
0 events. For a better experience and more opportunities
they are asked to complete their profile with a more detailed form, including these from
their hometown. This form is shown in \ref{fig:form} and described in next section.
\subsection{Registration form}
\begin{figure}[H]
    \centering
    \includegraphics[width=0.6\textwidth]{images/form.png}
    \caption{Form for the full registration}
    \label{fig:form}
\end{figure}
When user wants to upgrade to tier 1 they have to pay their subscription (as shown in \ref{fig:payment}).
The form for a full profile asks for many information, here just a few are displayed, but the 
real form will have more than 10 input boxes. 
\subsection{Payment page}
\begin{figure}[H]
    \centering
    \includegraphics[width=0.6\textwidth]{images/PaymentMethod.png}
    \caption{Payment page}
    \label{fig:payment}
\end{figure}
When a user pays his subscription, he will become a tier 1 user. He will be able to participate to more
events. Still he cannot create an event (only tier 3 or above users can do that). 
\subsection{Events list page}
\begin{figure}[H]
    \centering
    \includegraphics[width=0.6\textwidth]{images/EventList.png}
    \caption{Sign up page}
    \label{fig:events}
\end{figure}
Here all events al listed or filtered by tag. Users are allowed to see only events for their tier
or lower ones. Since all events has a location, we decided -according to ESN volunteers- to keep
events automatically filtered by tier.\\
Searching by tag or by cause are actually made by a REST API: using GET events are listed, administrator
can also use DELETE to delete an event. With a POST request, a new event can be created.\\
\subsection{Join event page}
\begin{figure}[H]
    \centering
    \includegraphics[width=0.6\textwidth]{images/JoinEvent.png}
    \caption{Sign up page}
    \label{fig:joinEvent}
\end{figure}
When a user wants to join an event, he can see all the details and the list of participants. Here
he can also see more detailed info about the event and confirm his participation. In case some events
will ask for a ticket one day, this page will redirect to a payment page, similar to \ref{fig:payment},
but linked to tickets for cinema, party etc.
\subsection{Change user info}
\begin{figure}[H]
    \centering
    \includegraphics[width=0.6\textwidth]{images/ChangeUserDetail.png}
    \caption{Change user's info}
    \label{fig:change info}
\end{figure}
Users can change their profile info. This page can be accessed by each user for his own profile or 
by an admin for any user.